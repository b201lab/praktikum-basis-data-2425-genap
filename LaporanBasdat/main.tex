\documentclass[a4paper,12pt]{report}

\usepackage[utf8]{inputenc}
\usepackage{graphicx}
\usepackage{setspace}
\usepackage{lipsum}
\usepackage[bahasa]{babel}
\usepackage{graphicx}
\usepackage{hyperref}
\usepackage{enumitem}
\usepackage{listings}
\usepackage{xcolor}
\usepackage{tcolorbox}
\usepackage{courier}
\usepackage{fancyhdr}
\usepackage{indentfirst}


% Package TikZ untuk diagram
\usepackage{tikz}
\usepackage{pgf}
\usetikzlibrary{mindmap}
\usetikzlibrary{arrows.meta}
\usetikzlibrary{positioning}
\usetikzlibrary{shapes,arrows}
\usetikzlibrary{er}
\usetikzlibrary{calc}

% Library tambahan untuk diagram kompleks
\usetikzlibrary{trees}
\usetikzlibrary{shadows}
\usetikzlibrary{backgrounds}
\usetikzlibrary{decorations}

% Atur penomoran section agar tidak terpengaruh chapter
% Ini akan memungkinkan penomoran section dimulai dari 1 di setiap chapter
\counterwithout{section}{chapter}

% Konfigurasi hyperref
\hypersetup{
    colorlinks=true,
    linkcolor=black,
    filecolor=magenta,      
    urlcolor=cyan,
    pdftitle={Modul Praktikum Sistem Manajemen Basis Data},
    pdfpagemode=FullScreen,
}

% Konfigurasi untuk kode program
\definecolor{codegreen}{rgb}{0,0.6,0}
\definecolor{codegray}{rgb}{0.5,0.5,0.5}
\definecolor{codepurple}{rgb}{0.58,0,0.82}
\definecolor{backcolour}{rgb}{0.95,0.95,0.92}

\lstdefinestyle{mystyle}{
    backgroundcolor=\color{backcolour},
    commentstyle=\color{codegreen},
    keywordstyle=\color{magenta},
    numberstyle=\tiny\color{codegray},
    stringstyle=\color{codepurple},
    basicstyle=\ttfamily\small,
    breakatwhitespace=false,
    breaklines=true,
    captionpos=b,
    keepspaces=true,
    numbers=left,
    numbersep=5pt,
    showspaces=false,
    showstringspaces=false,
    showtabs=false,
    tabsize=2
}

\lstset{style=mystyle}

% Konfigurasi header dan footer
\pagestyle{fancy}
\fancyhf{}
\fancyhead[L]{}
\fancyhead[R]{Departemen Teknik Komputer ITS}
\fancyfoot[C]{\thepage}
\renewcommand{\headrulewidth}{0.4pt}
\renewcommand{\footrulewidth}{0.4pt}

\title{Laporan Praktikum\\
Basis Data}
\author{}
\date{}


\begin{document}

% Halaman Cover
\begin{titlepage}
    \centering

    \includegraphics[width=0.3\textwidth]{images/logo-tekkom.png}\\[1cm]

    {\Huge \textbf{Laporan Praktikum}\\[0.3cm]}
    {\Huge \textbf{Basis Data}}\\[1cm]
    
    {\Large \textit{JUDUL MODUL 1}}\\[3cm]
    
    {\large \textit{Departemen Teknik Komputer}}\\
    {\large \textit{Institut Teknologi Sepuluh Nopember}}\\
    {\large \textit{Surabaya}}\\[3cm]

    \hfill Penulis: lab b201\\
    \hfill 2025
\end{titlepage}

\tableofcontents
\clearpage

%Modul
\chapter*{Modul}
\addcontentsline{toc}{chapter}{Modul}

%Isi modul disini

\lipsum[1-4] %Hapus jika ada modul

%Dasar Teori
\chapter*{Dasar Teori}
\addcontentsline{toc}{chapter}{Dasar Teori}

%Isi Dasar Teori disini   

\lipsum[5-6] %Hapus jika ada modul

%Hasil Praktikum
\chapter*{Hasil Praktikum}
\addcontentsline{toc}{chapter}{Hasil Praktikum}

%Isi dengan Hasil Praktikum

%Jika ingin memasukkan gambar, masukkan gambar ke dalam folder images dan sesuaikan nama file
\begin{figure}
    \centering
    \includegraphics[width=0.5\textwidth]{images/logo-tekkom.png} %ubah dengan gambar hasil praktikum dan width yang sesuai
    \caption{Contoh Hasil Praktikum} %ubah dengan caption yang sesuai
    \label{fig:hasil_praktikum} %ubah dengan label yang sesuai
\end{figure}



\lipsum[1-2] %Hapus jika ada isi Hasil Praktikum



%Tugas Modul

\chapter*{Tugas Modul}
\addcontentsline{toc}{chapter}{Tugas Modul}

\begin{enumerate}
    \item Bandingkan performa dengan dan tanpa index untuk setiap query \vspace{5px} %Isi sesuai dengan pertanyaan modul \newline 
    \lipsum[3] %Isi dengan jawaban dari pertanyaann modul
    \item \lipsum[4] %ganti dengan pertanyaan modul selanjutnya
\end{enumerate}

%Tugas Asistensi
%Jika tidak ada, maka bisa dihapus seluruhnya

\chapter*{Tugas Asistensi}
\addcontentsline{toc}{chapter}{Tugas Asistensi}

\begin{enumerate}
    \item Bla bla bla bla \vspace{5px} %Isi dengan soal asistensi \newline 
    \lipsum[5] %Isi dengan jawaban soal asistensi
    \item \lipsum[6] %ganti dengan soal selanjutnya
\end{enumerate} %Jika tidak ada Tugas Asistensi, maka bisa dihapus

%Kesimpulan
\chapter*{Kesimpulan}
\addcontentsline{toc}{chapter}{Kesimpulan}

%Isi dengan Kesimpulan Praktikum

\lipsum[6-7] %Hapus jika ada kesimpulan praktikum

\end{document}
